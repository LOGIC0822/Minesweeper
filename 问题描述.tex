\documentclass{article}
\usepackage{graphicx} % Required for inserting images
\usepackage[a4paper,margin=3cm]{geometry}
\usepackage{CJKutf8}
\usepackage{indentfirst}

\title{问题求解与实践课程大作业2问题描述}
\author{邴乙恒 522031910191}
\date{December 2023}

\begin{document}
\begin{CJK}{UTF8}{gbsn}

\maketitle

\section{问题描述}

本项目利用了fltk库实现了一个简易的“扫雷”游戏,支持玩家选择游戏难度或自定义难度,并包括绘制地图格阵,绘制“地雷”(黑点)和“旗帜”(红点),绘制点击过的点周围地雷数量,显示剩余地雷数量以及游戏进行时间等,基本实现了“扫雷”的各种功能。

\section{随机布雷}

本项目使用srand((unsigned)time(nullptr))来初始化随机数生成器。srand函数设置随机数生成器的种子,time(nullptr)返回当前的时间,用作种子。这样可以确保每次运行程序时,生成的随机数序列都不同。

此外,为了避免玩家第一次点击触雷,我们先检测一次左键点击事件,之后在除了被点击的格子之外的格子中随机布雷,然后计算出该格子周围的地雷数量显示在窗口中。

\section{周围无地雷的格子自动翻起}

为了加速游戏进程,“扫雷”加入了自动翻开周围无地雷的特性。要实现这个特性可以借助递归算法实现。但考虑到运行速度以及递归算法有可能引起的栈溢出的风险,本项目采取了广度优先搜索的非递归算法。

\section{左右键单击按钮的功能实现}

若按钮已经被打开过,左键单击没有效果,右键单击时会进行一个判断,若周围的地雷都已经被“旗帜”标记,则会翻开周围所有非地雷的格子。

若按钮未被打开,左键单击会进行一个判断,若是地雷,则会发出触雷警告并强制结束游戏(不会关闭窗口),否则正常翻开该格子,并显示周围地雷数(为0不显示),并会翻开周围其他周围没有地雷的格子(见第二点)。右键单击也会进行一个判断,若已经被“旗帜”标记过则没有效果,否则添加一个“旗帜”(红点)作为标记。

\section{计时器以及游戏时间显示}

计时器使用了fltk库中的Fl::add\_timeout函数,该函数会在指定的时间间隔后调用指定的函数。本项目中,计时器每隔1秒调用一次update\_time函数,该函数会更新窗口中的计时器显示的时间。当然,在游戏开始前计时器不会工作,结束后,计时器会被停止。

游戏在窗口下方设置了两个输出框,分别显示游戏进行的时间以及剩余的地雷数量,方便玩家进行推理。

\section{游戏难度}

游戏难度分为简单、中等、困难三种,分别对应10*10、16*16、30*16的格阵,地雷数量分别为10、40、99。玩家可以通过在游戏开始前点击相应的按钮选择游戏难度。此外,游戏也支持自定义格阵与地雷数量,玩家可以通过在游戏开始前点击自定义按钮,然后在弹出的窗口中输入自定义的格阵与地雷数量,点击确定后开始游戏。

\section{游戏结束标志}

第四点提到过,若玩家触雷,则会强制结束游戏。结束游戏后会显示所有地雷的位置以及被错误标记的地雷位置。

此外,若所有非地雷格子都已经被翻开,则无论地雷是否被标记,玩家均宣告胜利,游戏也随之结束。

在一局游戏结束后,玩家可以选择关闭窗口退出游戏或者点击重置按钮返回游戏开始状态,重新选择难度开始游戏。


\end{CJK}
\end{document}